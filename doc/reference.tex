\documentclass{article}
\usepackage{amsmath}

% page size
\setlength{\hoffset}{-1in}
\setlength{\voffset}{-1in}
\setlength{\oddsidemargin}{3.5cm}
\setlength{\evensidemargin}{3.5cm}
\setlength{\topmargin}{2cm}
\setlength{\textwidth}{14cm}
\setlength{\textheight}{22cm}

\newcommand{\ct}{\texttt}
\newcommand{\NULL}{\ct{NULL}}
\setlength{\parindent}{0pt}
\newcommand{\param}[1]{\vspace{2em}\ct{#1}\nopagebreak\\[0.5em]\nopagebreak}
\newcommand{\default}[1]{\\Default: #1}
\newcommand{\rvec}[1]{\left(#1\right)}

\newcommand{\obj}{\ct{obj}}
\newcommand{\rhs}{\ct{rhs}}
\newcommand{\lb}{\ct{lb}}
\newcommand{\ub}{\ct{ub}}
\newcommand{\numconstr}{\ct{num\_constr}}
\newcommand{\numvar}{\ct{num\_var}}
\newcommand{\x}{\ct{x}}
\newcommand{\slack}{\ct{slack}}
\newcommand{\xl}{\ct{xl}}
\newcommand{\xu}{\ct{xu}}
\newcommand{\y}{\ct{y}}
\newcommand{\z}{\ct{z}}
\newcommand{\zl}{\ct{zl}}
\newcommand{\zu}{\ct{zu}}
\newcommand{\vbasis}{\ct{vbasis}}
\newcommand{\cbasis}{\ct{cbasis}}
\newcommand{\basic}{\ct{IPX\_basic}}
\newcommand{\nonbasic}{\ct{IPX\_nonbasic}}
\newcommand{\nonbasiclb}{\ct{IPX\_nonbasic\_lb}}
\newcommand{\nonbasicub}{\ct{IPX\_nonbasic\_ub}}
\newcommand{\superbasic}{\ct{IPX\_superbasic}}

\title{IPX Reference}
\author{Version 1.0}

\begin{document}
\maketitle

%-------------------------------------------------------------------------------
\section*{Functionality}
%-------------------------------------------------------------------------------

IPX solves linear programming (LP) problems in the form
\begin{subequations}
  \begin{alignat}{2}
    \underset{\x}{\text{minimize}} &\quad& &\obj^T\x \\
    \label{eq:ax=b}
    \text{subject to} && &A\x\{\ge,\le,=\}\rhs, \\
    && &\lb\le\x\le\ub.\
  \end{alignat}
\end{subequations}
The matrix $A$ has \numconstr\ rows and \numvar\ columns. Associated with
\eqref{eq:ax=b} are dual variables \y\ with the sign convention that
\begin{subequations}
  \label{eq:signy}
  \begin{alignat}{2}
    \y[i]\ge0 &\quad& &\text{if constraint is of type $\ge$,} \\
    \y[i]\le0 && &\text{if constraint is of type $\le$,} \\
    \y[i]\text{ free} && &\text{if constraint is of type $=$.}
  \end{alignat}
\end{subequations}
Associated with $\lb\le\x$ and $\x\le\ub$ are dual variable $\zl\ge0$ and
$\zu\ge0$ respectively. Entries of $-\lb$ and $\ub$ can be infinity, in which
case the dual is fixed at zero.

\subsubsection*{Interior Point Method}
The interior point method (IPM) computes a primal-dual point
$\rvec{\x,\slack,\xl,\xu,\y,\zl,\zu}$ that approximately satisfies
\begin{subequations}
  \label{eq:res}
  \begin{align}
    \label{eq:pres}
    &A\x+\slack=\rhs, \quad \x-\xl=\lb, \quad \x+\xu=\ub,\\
    \label{eq:dres}
    &A^T\y+\zl-\zu=\obj,
  \end{align}
\end{subequations}
and that is guaranteed to satisfy $\xl\ge0$, $\xu\ge0$, \eqref{eq:signy} and
\begin{subequations}
  \label{eq:signslack}
  \begin{alignat}{2}
    \slack[i]\le0 &\quad& &\text{if constraint is of type $\ge$,} \\
    \slack[i]\ge0 && &\text{if constraint is of type $\le$,} \\
    \slack[i]=0 && &\text{if constraint is of type $=$.}
  \end{alignat}
\end{subequations}
In theory, the IPM iterates will in the limit satisfy \eqref{eq:pres} and
\eqref{eq:dres}, and the primal objective will equal the dual objective
\begin{equation}
  \label{eq:dobj}
  \rhs^T\y + \lb^T\zl - \ub^T\zu.
\end{equation}
(Entries for which -\lb\ or \ub\ is infinity are understood to be dropped from
the sum.)

\subsubsection*{Crossover}
The crossover method recovers an optimal basis from the interior solution. A
basis is defined by variable and constraint statuses
\begin{align}
  \vbasis[j] &\in \{\basic, \nonbasiclb, \nonbasicub, \superbasic\}, \\
  \cbasis[i] &\in \{\basic, \nonbasic\}.
\end{align}
The columns of $A$ for which $\vbasis[j]=\basic$ and the columns of the identity
matrix for which $\cbasis[i]=\basic$ form a square, nonsingular matrix of
dimension \numconstr. The corresponding basic solution $\rvec{\x,\slack,\y,\z}$
is obtained by setting
\begin{subequations}
  \begin{alignat}{3}
    \z[j]&=0& &\quad& &\text{if $\vbasis[j]=\basic$}, \\
    \x[j]&=\lb[j]& && &\text{if $\vbasis[j]=\nonbasiclb$}, \\
    \x[j]&=\ub[j]& && &\text{if $\vbasis[j]=\nonbasicub$}, \\
    \x[j]&=0& &&      &\text{if $\vbasis[j]=\superbasic$}, \\
    \y[i]&=0& &&      &\text{if $\cbasis[i]=\basic$}, \\
    \slack[i]&=0& &&  &\text{if $\cbasis[i]=\nonbasic$}
  \end{alignat}
\end{subequations}
and computing the remaining components such that $A\x+\slack=\rhs$ and
$A^T\y+\z=\obj$. The basis is primal feasible if $\lb\le\x\le\ub$ and
\eqref{eq:signslack} hold; the basis is dual feasible if \eqref{eq:signy} holds
and
\begin{subequations}
  \begin{alignat}{2}
    \z[j]\ge0 &\quad& &\text{if $\vbasis[j]=\nonbasiclb$}, \\
    \z[j]\le0 &\quad& &\text{if $\vbasis[j]=\nonbasicub$}, \\
    \z[j]=0 &\quad& &\text{if $\vbasis[j]=\superbasic$}.
  \end{alignat}
\end{subequations}
The IPX crossover consists of a primal and dual push phase. Depending on the
accuracy of the interior solution and the numerical stability of the LP problem,
the obtained basis may not be primal and/or dual feasible. In this case
reoptimization with an external simplex code is required.

\end{document}
